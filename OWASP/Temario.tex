\documentclass[12pt, letter-paper]{article}
\usepackage[inner=1.5cm,outer=1.5cm,top=2.5cm,bottom=2.5cm]{geometry}
\pagestyle{empty}
\usepackage{graphicx}
\usepackage{fancyhdr, lastpage, bbding, pmboxdraw}
\usepackage[usenames,dvipsnames]{color}
\usepackage[utf8]{inputenc}
\definecolor{darkblue}{rgb}{0,0,.6}
\definecolor{darkred}{rgb}{.7,0,0}
\definecolor{darkgreen}{rgb}{0,.6,0}
\definecolor{red}{rgb}{.98,0,0}
\usepackage[colorlinks,pagebackref,pdfusetitle,urlcolor=darkblue,citecolor=darkblue,linkcolor=darkred,bookmarksnumbered,plainpages=false]{hyperref}
\renewcommand{\thefootnote}{\fnsymbol{footnote}}

\pagestyle{fancyplain}
\fancyhf{}
\lhead{ \fancyplain{}{OWASP Top 10} }
\rhead{ \fancyplain{}{Marzo 2018} }
\fancyfoot[RO, LE] {page \thepage\ of \pageref{LastPage} }
\thispagestyle{plain}

\usepackage{listings}
\usepackage{caption}
\DeclareCaptionFont{white}{\color{white}}
\DeclareCaptionFormat{listing}{\colorbox{gray}{\parbox{\textwidth}{#1#2#3}}}
\captionsetup[lstlisting]{format=listing,labelfont=white,textfont=white}
\usepackage{verbatim}
\usepackage{fancyvrb}
\usepackage{acronym}
\usepackage{amsthm}
\VerbatimFootnotes 

\definecolor{OliveGreen}{cmyk}{0.64,0,0.95,0.40}
\definecolor{CadetBlue}{cmyk}{0.62,0.57,0.23,0}
\definecolor{lightlightgray}{gray}{0.93}

\lstset{
  basicstyle=\ttfamily,                   % Code font, Examples: \footnotesize, \ttfamily
  keywordstyle=\color{OliveGreen},        % Keywords font ('*' = uppercase)
  commentstyle=\color{gray},              % Comments font
  numbers=left,                           % Line nums position
  numberstyle=\tiny,                      % Line-numbers fonts
  stepnumber=1,                           % Step between two line-numbers
  numbersep=5pt,                          % How far are line-numbers from code
  backgroundcolor=\color{lightlightgray}, % Choose background color
  frame=none,                             % A frame around the code
  tabsize=2,                              % Default tab size
  captionpos=t,                           % Caption-position = bottom
  breaklines=true,                        % Automatic line breaking?
  breakatwhitespace=false,                % Automatic breaks only at whitespace?
  showspaces=false,                       % Dont make spaces visible
  showtabs=false,                         % Dont make tabls visible
  columns=flexible,                       % Column format
  morekeywords={__global__, __device__},  % CUDA specific keywords
}

\begin{document}
\begin{center}
  {\Large \textsc{OWASP Top 10}}
\end{center}
\begin{center}
  The Ten Most Critical Web Application Security Riskssss
\end{center}

\begin{center}
  \rule{6in}{0.4pt}
  \begin{minipage}[t]{.75\textwidth}
    \begin{tabular}{llcccll}
      \textbf{Instructora:} & Andrea Gómez & & &
      \textbf{Fecha:} & Marzo, 2018
    \end{tabular}
  \end{minipage}
  \rule{6in}{0.4pt}
\end{center}
\vspace{.5cm}
\setlength{\unitlength}{1in}
\renewcommand{\arraystretch}{2}

\noindent\textbf{Temario:}
\begin{enumerate}

\item Injection
  \begin{enumerate}
  \item Item 1
    \begin{enumerate}
    \item Subitem 1
    \item Subitem 2
    \end{enumerate}
  \item Item 2
    \begin{enumerate}
    \item Subitem 1
    \item Subitem 2
    \item Subitem 3
    \item Subitem 4
    \end{enumerate}
  \end{enumerate}

\item Broken Authentication
  \begin{enumerate}
  \item Item 1
    \begin{enumerate}
    \item Subitem 1
    \item Subitem 2
    \end{enumerate}
  \item Item 2
    \begin{enumerate}
    \item Subitem 1
    \item Subitem 2
    \item Subitem 3
    \item Subitem 4
    \end{enumerate}
  \end{enumerate}

\item XSS
  \begin{enumerate}
  \item Item 1
    \begin{enumerate}
    \item Subitem 1
    \item Subitem 2
    \end{enumerate}
  \item Item 2
    \begin{enumerate}
    \item Subitem 1
    \item Subitem 2
    \item Subitem 3
    \item Subitem 4
    \end{enumerate}
  \end{enumerate}

\item CSRF
  \begin{enumerate}
  \item Item 1
    \begin{enumerate}
    \item Subitem 1
    \item Subitem 2
    \end{enumerate}
  \item Item 2
    \begin{enumerate}
    \item Subitem 1
    \item Subitem 2
    \item Subitem 3
    \item Subitem 4
    \end{enumerate}
  \end{enumerate}

\item CSRF
  \begin{enumerate}
  \item Item 1
    \begin{enumerate}
    \item Subitem 1
    \item Subitem 2
    \end{enumerate}
  \item Item 2
    \begin{enumerate}
    \item Subitem 1
    \item Subitem 2
    \item Subitem 3
    \item Subitem 4
    \end{enumerate}
  \end{enumerate}

\item Broken Access Controls
  \begin{enumerate}
  \item Item 1
    \begin{enumerate}
    \item Subitem 1
    \item Subitem 2
    \end{enumerate}
  \item Item 2
    \begin{enumerate}
    \item Subitem 1
    \item Subitem 2
    \item Subitem 3
    \item Subitem 4
    \end{enumerate}
  \end{enumerate}

\item Sensitive Data Exposure
  \begin{enumerate}
  \item Item 1
    \begin{enumerate}
    \item Subitem 1
    \item Subitem 2
    \end{enumerate}
  \item Item 2
    \begin{enumerate}
    \item Subitem 1
    \item Subitem 2
    \item Subitem 3
    \item Subitem 4
    \end{enumerate}
  \end{enumerate}

\item Insecure Direct Object References
  \begin{enumerate}
  \item Item 1
    \begin{enumerate}
    \item Subitem 1
    \item Subitem 2
    \end{enumerate}
  \item Item 2
    \begin{enumerate}
    \item Subitem 1
    \item Subitem 2
    \item Subitem 3
    \item Subitem 4
    \end{enumerate}
  \end{enumerate}

\item Misconfiguration
  \begin{enumerate}
  \item Item 1
    \begin{enumerate}
    \item Subitem 1
    \item Subitem 2
    \end{enumerate}
  \item Item 2
    \begin{enumerate}
    \item Subitem 1
    \item Subitem 2
    \item Subitem 3
    \item Subitem 4
    \end{enumerate}
  \end{enumerate}

\item Insecure Components
  \begin{enumerate}
  \item Item 1
    \begin{enumerate}
    \item Subitem 1
    \item Subitem 2
    \end{enumerate}
  \item Item 2
    \begin{enumerate}
    \item Subitem 1
    \item Subitem 2
    \item Subitem 3
    \item Subitem 4
    \end{enumerate}
  \end{enumerate}

\item Redirects
  \begin{enumerate}
  \item Item 1
    \begin{enumerate}
    \item Subitem 1
    \item Subitem 2
    \end{enumerate}
  \item Item 2
    \begin{enumerate}
    \item Subitem 1
    \item Subitem 2
    \item Subitem 3
    \item Subitem 4
    \end{enumerate}
  \end{enumerate}
\end{enumerate}

\vskip.15in
\noindent \textbf{Instalando lo necesario en Arch Linux:} Recomiendo encarecidamente llevar instalados los siguientes paquetes antes de comenzar el curso para poder hacer un uso eficiente del tiempo. Si no usan Arch Linux, bastará con adecuar los comandos al gestor de paquetes usado por su distribución preferida:

\begin{lstlisting}
  $ sudo pacman -S ghc
\end{lstlisting}

Si bien GHC es todo lo que necesitaremos pues el curso es introductorio, recomiendo instalar los siguientes paquetes en adición a GHC para quien desee adentrarse al hermoso mundo de Haskell.

\begin{lstlisting}
  $ sudo pacman -S cabal-install haskell-haddock-api haskell-haddock-library happy alex emacs 
\end{lstlisting}

Recomiendo también los siguientes paquetes para Emacs pues enriquecerán el ya increíble editor de texto Emacs. Solo requieren tener habilitado el repositorio Melpa para instalarlos dentro de Emacs y posteriormente, editar el archivo .emacs para inicializarlos.

\begin{lstlisting}
  M-x package-install RET haskell-mode
  M-x package-install RET auto-complete
  M-x package-install RET ac-haskell-process
\end{lstlisting}

\end{document} 
